\documentclass[oneside]{memoir}

\usepackage{lmodern}
\usepackage[T1]{fontenc}
\usepackage[spanish,activeacute]{babel}
\usepackage{mathtools}
\usepackage{graphicx}

\title{YourPlacesBot - A Telegram Bot}
\author{David Quesada L\'opez y Mateo Garc\'ia Fuentes}

\newcommand{\romanpages}{
\pagenumbering{roman}
\thispagestyle{empty}
\beforepartskip
\centering

\thetitle

\theauthor

GRADO EN INGENIER\'IA INFORM\'ATICA 

FACULTAD DE INGENIER\'IA INFORM\'ATICA

UNIVERSIDAD COMPLUTENSE DE MADRID

\includegraphics{logo.jpg}

TRABAJO DE FIN DE GRADO EN INGENIER\'IA INFORM\'ATICA

Director: Carlos Gregorio Rodríguez

\today
\afterpartskip
\newpage
\thispagestyle{empty}
\beforepartskip
\raggedright

\section{Autorizaci\'on de difusi\'on y utilizaci\'on}
\afterpartskip
\newpage
\thispagestyle{empty}
\beforepartskip
\raggedright

\section{Agradecimientos}
Gracias a Nick Lee (https://github.com/nickoala) por desarrollar la API de Python para bots de Telegram y usar una licencia MIT.

\afterpartskip
\newpage
}

\newcommand{\indexpage}{
\frontmatter
\setcounter{page}{4}
\pagestyle{plain}

\section{\'Indice}
\tableofcontents

\newpage
}


\begin{document}
\romanpages
\indexpage

\mainmatter

\newpage
\section{\'Indice de figuras}

\newpage
\section{\'Indice de abreviaturas}

\newpage
\section{Resumen}

Palabras clave:

\newpage
\section{Abstract}

Keywords: 

\newpage
\section{Cap\'itulo 1: Introducci\'on}

\newpage
\section{Cap\'itulo 2: Materiales y m\'etodos}

\newpage
\section{Cap\'itulo 3: Resultados}

\newpage
\section{Cap\'itulo n: Conclusiones}

\newpage
\section{Ap\'endice}

\newpage
\section{Bibliograf\'ia}

\newpage
\section{Anexo}

\newpage
\section{Glosario}

\end{document}
