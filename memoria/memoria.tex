\documentclass[oneside]{memoir}

\usepackage{lmodern}
\usepackage[T1]{fontenc}
\usepackage[spanish,activeacute]{babel}
\usepackage{mathtools}
\usepackage{graphicx}
\usepackage{titlesec}

\titleformat{\chapter}[display]
  {\normalfont\bfseries}{}{0pt}{}

\title{YourPlacesBot - A Telegram Bot}
\author{David Quesada L\'opez y Mateo Garc\'ia Fuentes}

\newcommand{\romanpages}{
\pagenumbering{roman}
\thispagestyle{empty}
\beforepartskip
\centering

\thetitle

\theauthor

GRADO EN INGENIER\'IA INFORM\'ATICA 

FACULTAD DE INGENIER\'IA INFORM\'ATICA

UNIVERSIDAD COMPLUTENSE DE MADRID

\includegraphics{logo.jpg}

TRABAJO DE FIN DE GRADO EN INGENIER\'IA INFORM\'ATICA

Director: Carlos Gregorio Rodr�guez

\today
\afterpartskip
\newpage

\thispagestyle{empty}
%\beforepartskip
\raggedright
\section{Autorizaci\'on de difusi\'on y utilizaci\'on}
%\chapter[Autorizaci\'on de difusi\'on y utilizaci\'on]{Autorizaci\'on de difusi\'on y utilizaci\'on}
David Quesada L\'opez \newline
\newline
\newline
\newline
\today \newline
\newline
Mateo Garc\'ia Fuentes \newline
\newline
\newline
\newline
\today \newline

\afterpartskip
\newpage
\thispagestyle{empty}
%\beforepartskip
\raggedright
\section{Agradecimientos}
%\chapter[Agradecimientos]{Agradecimientos}

Gracias a Nick Lee (https://github.com/nickoala) por desarrollar telepot, un framework de Python para API de Telegram Bot y desarrollarlo bajo una licencia MIT.

\afterpartskip
\newpage
}

\newcommand{\indexpage}{
\frontmatter
\setcounter{page}{4}
\pagestyle{plain}

\tableofcontents

\newpage
}


\begin{document}
\romanpages
\indexpage

\mainmatter

\newpage
\chapter[\'Indice de figuras]{\'Indice de figuras}

\newpage
\chapter[\'Indice de abreviaturas]{\'Indice de abreviaturas}

\newpage
\chapter[Resumen]{Resumen}

Palabras clave:

\newpage
\chapter[Abstract]{Abstract}

Keywords: 

\newpage
\chapter[Cap\'itulo 1: Introducci\'on]{Cap\'itulo 1: Introducci\'on}

\newpage
\chapter[Cap\'itulo 2: Materiales y m\'etodos]{Cap\'itulo 2: Materiales y m\'etodos}

\newpage
\chapter[Cap\'itulo 3: Resultados]{Cap\'itulo 3: Resultados}

\newpage
\chapter[Cap\'itulo n: Conclusiones]{Cap\'itulo n: Conclusiones}

\newpage
\chapter[Ap\'endice]{Ap\'endice}

\newpage
\begin{thebibliography}{}
%Falta escribirlo en formato IEEE
\bibitem
*LaTex -> http://texdoc.net/texmf-dist/doc/latex/memoir/memman.pdf 
\bibitem
*Python -> https://docs.python.org/2.7/
\bibitem
*API Google Maps Docs -> http://googlemaps.github.io/google-maps-services-python/docs/2.4.5/
\bibitem
*Telegram Bots -> https://core.telegram.org/bots
\bibitem
*API Telepot -> http://telepot.readthedocs.io/en/latest/
\end{thebibliography}

\newpage
\chapter[Anexo]{Anexo}

\newpage
\chapter[Glosario]{Glosario}

\end{document}